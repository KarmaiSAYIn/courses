\documentclass{article}
\usepackage{titling}
\usepackage{setspace}
\usepackage{changepage}
\usepackage{lipsum}
\usepackage[margin=1in]{geometry}

\setlength\paperwidth{8.5in}
\setlength\paperheight{11in}
\setlength\parindent{24pt}

\renewcommand{\maketitle}
{
    \begin{center}
        {\huge\bfseries\thetitle}\\
        \vspace{1ex}
        by \theauthor\\
        \thedate\\
    \end{center}
    \hrule
}

\title{Positive vs Negative Thinking}
\author{Jonah Mondargon}
\date{\today}

\begin{document}

\pagestyle{empty}
\maketitle

\doublespacing

\subsection*{What type of mindset do you have 80/20, or 90/10 and why?}
    \begin{adjustwidth}{24pt}{0in}
        The 90/10 rule is my goto; it adds that extra 10\% to the responsibility of how I react; and it's the notion used by
        computer scientists saying ``90\% of the runtime comes from 10\% of the code.'' This way, I know I have all of the
        control, quite literally; I walk around with an A.
    \end{adjustwidth}
\subsection*{Do you only think positive when things are going your way (Front Runner)?}
    \begin{adjustwidth}{24pt}{0in}
        I do not think positive only when things are going my way. Without the ability to be positive in the opposite situation,
        you'll never get out of it. Things actually rarely go your way, when it does it's important to remain in the same mindset
        as before, otherwise you run the risk of avoiding future occurrences.
    \end{adjustwidth}
\subsection*{How can you adopt the mindset of being a morning person and apply it to school?}
    \begin{adjustwidth}{24pt}{0in}
        I'm already a morning person, getting to school (or joining the conferences...) is a simpler matter of putting on your 
        school clothes an hour early. Having an oppourtunity to do something that pleases you before something burdensome like
        school.
    \end{adjustwidth}
\subsection*{How can you recognize yourself having negative feelings/thoughts, and how can you change that mindset?}
    \begin{adjustwidth}{24pt}{0in}
        Three words ``Is this useful?''
    \end{adjustwidth}

\end{document}
