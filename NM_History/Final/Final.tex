\documentclass[12pt]{article}
\usepackage{titling}
\usepackage{lipsum}
\usepackage[hyphens]{url}
\usepackage{hyperref}
\usepackage{geometry}

\setlength\paperwidth{8.5in}
\setlength\paperheight{11in}
\setlength\parindent{24pt}

\newcommand{\PutTitle}[1]
{
    \begin{center}
        {\huge\bfseries\thetitle}\\
        by \theauthor\\
        \thedate\\
        #1        
    \end{center}
    \hrule
    \vspace{2ex}
}

\hypersetup
{
    colorlinks=true,
    linkcolor=blue,
    urlcolor=blue,
}

\begin{document}

\title{Final Exam}
\author{Jonah Mondragon}
\date{\today}
\PutTitle{New Mexico History Period 3}
\pagestyle{headings}

\section*{5 Important Facts About New Mexico}
These are five facts about New Mexico that aren't necessarily imperative to the dwelling of an individual within, but are relevant to New Mexico's conceptual presence as a place of historical and cultural significance.
I'll list them in no particular order with a description behind each. 

\begin{enumerate}
    \item{The name ``New Mexico'' isn't named after the modern country ``Mexico'' but after the {\color{blue}\underline{\href{https://www.sjsu.edu/faculty/watkins/aztecs.htm}{``Valley of Mexico''}}}}
        \paragraph~
        The Aztec's are, to this day, considered to have been an advanced civilization; part of this meant having the ability to set religious dogma explaining phenomena.
        As a sort of rule, religious beliefs inspire works of art relating to them, the Aztecs made fine art.
        
\end{enumerate}

%\section*{The Future of New Mexico}

%\section*{Regarding the Gorge Bridge}

%\section*{Regarding My Grade for This Semester}

%\section*{Regarding the School Methodology}

\newpage
\section*{References}
\begin{enumerate}
    \item{\url{http://rubens.anu.edu.au/raid1/student_projects97/aztec/AztecHistory.html/History1.html}}
\end{enumerate}
\section*{Hyperlinks}
\begin{enumerate}
    \item{\url{https://www.sjsu.edu/faculty/watkins/aztecs.htm}}
\end{enumerate}

\end{document}
