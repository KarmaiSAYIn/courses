\documentclass[12pt]{article}
\usepackage{gensymb}
\usepackage{titling}
\usepackage{setspace}
\usepackage{hyperref}
\usepackage{lipsum}
\usepackage{fancyhdr}
\usepackage[margin=1in]{geometry}

\setlength\paperwidth{8.5in}
\setlength\paperheight{11in}
\setlength\parindent{24pt}

\hypersetup
{
    colorlinks=true,
    linkcolor=blue,
    urlcolor=blue,
}

\newcommand{\PutTitle}[1]
{
    \begin{center}
        {\huge\bfseries\thetitle}\\
        by \theauthor\\
        \thedate\\
        #1        
    \end{center}
    \hrule
    \vspace{2ex}
}

\begin{document}

\doublespacing
\thispagestyle{empty}

\title{Lowrider Description}
\author{Jonah Mondragon}
\date{\today}

\PutTitle{Physics Period 6}

My choice for the design of a stereotypical lowrider to present my idea for the lowriders of the future
    stems from my belief that the "lowrider" as we know it today will never change, at least with the
    type of car we use to make the transformation from a roadster. 
    The reason I make this conjecture is because of the culture that has surrounded lowriders since 
    their inception, in my mind, and with a completly subjective statement, cars made today can never 
    hold a candle the status of current lowriders.
    The lowrider in my drawing has two fins above the rear lights, similar to a
    {\color{blue}\underline{\href{http://cascadiaclassics.com/home/2018/1/20/1959-chevrolet-el-camino}{1959 Chevrolet El Camino}}}
    but without the dramatic V-shape on the rear.
The interior, though I haven't colored the image, are medium sandalwood colored plushy-cushioned seats,
    sitting on them gives you a spring sensation, maneuvering your weight in a downward direction will
    only result in an immediate retreat upwards.
Even though the shell will remain the same, the mechanical elements will definitely improve, in this
    particular future-thought lowrider the hydraulic system incorporates a booster not dissimilar to
    Iron-Man's thruster boots, allowing the car to do a 360$^{\circ}$ midair barrel roll, in any 
    direction and land on it's wheels.
Some precedence exists that would suggest the ability to take flight in some instances, such as one's
    involving the danger of the passengers, such as a great fall off a cliff; further suggesting an
    implicit sentient presence in the vehicle, but that's just a byproduct of the times I'm envisioning
    for this car.

\end{document}
