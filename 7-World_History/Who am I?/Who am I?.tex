\documentclass[12pt]{article}
\usepackage{titling}
\usepackage{setspace}
\usepackage{hyperref}
\usepackage{lipsum}
\usepackage[margin=1in]{geometry}

\setlength\paperwidth{8.5in}
\setlength\paperheight{11in}
\setlength\parindent{24pt}

\renewcommand\thesubsection{\arabic{subsection}}
\hypersetup
{
    colorlinks=true,
    linkcolor=blue,
    urlcolor=blue,
}

\begin{document}

\doublespacing

\begin{titlepage}
    \begin{center}
        \vspace*{1.5in}
        {\huge\bfseries{Who Am I? DRAFT \#1}}\\
            {\bfseries{by Jonah Mondragon}}\\
            World History Period 7\\
            \today
    \end{center}
\end{titlepage}

\section*{Part A: All About Me}

I was born on November 2, 2005 at Taos Holy Cross Hospital.  I had managed to
wrap my neck in the umbilical cord such that I had to be removed from my
mother's stomach by c-section.  Part of my inception involved a custody battle
between my mother and father.  This custody battle forced my father to sell a
1968 Chevrolet Camaro Super Sport in order to pay for his lawyer, I would be
driving it around today had he not sold it.  Long story short, my father won
custody over me and I have lived out the first fifteen years of my life in a
large house next to the {\color{blue}
\underline{\href{https://www.taoscountryclub.com/}{Taos Country Club}}} with
him. My dad's name is Ernie Mondragon and my mom's name is Lucy Robertson.

I bear crawled before I crawled normally and, according to my father, ran
before I walked.  In the courtroom where my parents allowed their lawyers to
fight for the right of raising me, it was the first time I'd been able to see
my father in months; <more precise?> I heard his voice and ran down the
courtroom walkway to him as soon as I was out of my mother's hold, I was three
years old. <fact check this> The reason for this custody battle boils down to
the fact that my mother ``kidnapped'' me in the attempt to extricate me from
the ``abusive'' hands of my father (the exact details of this don't have the
need to be related here, but know that all is fine with my situation now).  My
mother obtained possession of me without the legal authority to do so a handful
of times, the exact chronological order of these events is vague at best and
the only way to decide of it exactly would be to scrounge through court
transcripts and other documents, an endeavor that I don't want to partake in as
I have exerted enough mental energy throughout my lifetime being concerned with
my early childhood.

The initial arrangement of my custody situation was one that most kids in a
situation similar to mine have had, shared custody; my father would have me for
a week, then my mother would have me for a week (perhaps it was some other
period of time, I can't remember).  Of the time spent with my father I have
quite literally no recollection, they were spent either with my father's ex
wife, a woman whom I've grown to know as my grandma or playing video games; I
don't remember going to school, it must have been before I was enrolled in
anything beyond a daycare, again my memories overlap quite a bit
chronologically.  Come to think of it my mother and father have never married.

A woman named Dawn (I believe ``Dawn'' is short for ``Dawna,'' I know, quite
the truncation) at one point was my mother's landlord.  Dawn had a treehouse in
the backyard on an old tree, I believe it's a Chinese elm, the treehouse was
removed the last time I saw the property, but I didn't get the best view so I
may have just not seen it.  One time inside of the treehouse, on my own, there
was a wasp, and I guess the wasp sensed my fear of it because it stung me
square on the nose, it was the first time I remember being stung; I descended
the latter with a blister on the tip of my nose to my mother and Dawn, when I
entered the threshold of the door with tears dripping down my face my mother
and Dawn got honey and put it on my nose.  For some reason I associate this
memory with bears; I believe I asked about why bears like honey or something,
this association likely came from
{\color{blue}\underline{\href{https://winniethepooh.disney.com/}{Winnie The
Pooh}}}.  The apartment my mom lived in was on the second floor of Dawn's
house, a rather small one, and I remember bringing toys of Batman and Shrek to
this apartment to and from my dad's house, this made the established boundary
between life with my father and life with my mother less defined.

Dawn had children, I believe two, and they had guinea pigs, likely around two.
When no one was watching, as I was allowed to play or pet them, I would throw
the guinea pigs into the air as far up as I could and catch them on the way
down, they made a squealing noise that I found humorous; but my mom caught me
doing this one time and told me I could end up hurting them.  Eventually I got
my own guinea pig, far later in my life as my mother was no longer living in
the country, and still not seeing how I could hurt it I threw it up into the
air the same way as the guinea pigs belonging to Dawn's kids.  I don't remember
if at one time I didn't catch it, it had a heart attack or it simply stopped
breathing, but it died because of what I was doing, I placed the poor thing
into it's cage and pretended to sleep; I still hold back tears a little bit as
I remember my dad looking into the cage to try and play with it (I don't
remember its gender or its name) and noticing that it wasn't breathing.

My first shot with a gun was a bullseye (which happened sometime before or
around the age of seven), my dad got an empty soup can with a quality badge
printed on the front of it, and told me ``This is the bullseye, aim here.'' It
took me about ten seconds to line up the shot and after I fired we went up
close to inspect it; my dad had been surprised that I even hit it in the first
place as he couldn't because of the distance we were shooting from.  To some
respect, this event is proof that I'm a natural shooter, but if I am I haven't
honed that talent quite yet as I've seldom shot a gun after that.  The reason I
mention this isn't just to toot my own horn, though there is cadence to that
objective, but to demonstrate the derivation machinery of a lot of my early
childhood was influenced by, I truly believed I was above others, the teachers
who taught me, my friends who played with me, my family who guided me. The
reason I believed this is because of the special attention I would receive and
the notion I sensed to be present in the people around me, from as early as I
can remember being able to conversate, that I had a sort of special ability,
perhaps a particular aptness for understanding my environment. Now I realize
that everyone around me has tried to present my ``potential,'' as they say to
me; I believe that same potential exists in everybody, and perhaps I could do
something to help them realize it.

One of my memories with my mother is my in the backseat of her car, I was
around eight years old at the time and she wouldn't allow me to sit in the
front seat, she had just finished telling me that warm water is better for you
than cold water, but that drinking warm water from a plastic bottle isn't as
particles from the bottle end up in the water and therefore into your body,
when she withdrew from a freshly acquired {\color{blue}\underline{\href
{https://www.cidsfoodmarket.com/}{Cid's Food Market}}} shopping bag a container
of peculiar green meat cubes, beef, and she offered it to me. I loved the stuff
enough and in my inquisitive nature I asked her why it was green, I don't
remember what reason she gave for the color but she mentioned that it came from
a cow. This wasn't the first I'd heard about the farming of animals and I had
long since accepted the fact that other things would die so that I would live,
but for some reason, perhaps it was the color of the meat, or its flavor, or
the name and logo of the brand that sold it that got me in this specific
instance it hit me in a particular way and I started to cry; it seemed such a
cruel injustice that it would be so flavorful and yet in such vain that a cow
was killed in order to be sliced up into cubes and placed into a container for
me to eat. I'm not a vegetarian, but in that moment I was considering it.

After a certain point my father had full custody over me (perhaps he had full
custody over me ever since the initial court settlement, I don't actually
remember) but he allowed time for me to be with my mom, he recognized, and
continues to, the role a mother plays in one's life. There was a man named
Eliot that me and my mom would frequent, the best way I could describe him
would be as real life Mr.\ Miyagi, he's an elderly man who maintains great
physical fitness in his old age through daily physical exercise, using proven
techniques, and practicing meditation. My mom would come to him for guidance
and bring me with; I'll still visit him every so often for the same purpose.
The reason I mention Eliot is that she was the only friend of my mom's that I
was directly aware of, as in not simply in passing, and she requested of my
father that I and her spend a few days at a friend's house of whom I was
unfamiliar, I believe the exact number of days was two. This is not where we
ended up going. I don't know exactly what occurred, and I have not the
delegation to inquire, but we arrived to some destination, it was dark and all
I could see were a few cars parked outside a dimly lit porch, and I stayed in
the car longer than would have made sense given an overnight visit at a
friend's house; I fell asleep and when I woke up we were driving through an
unfamiliar road surrounded by sagebrush.  ``Where are we going?'' I asked, no
response. The next thing I remember is daytime, I woke up in the backseat and
at some point my mom asked me if I wanted to see an owl, I believe she gave it
a name and asked used the name instead of the word ``owl''; we got out and she
opened the truck, underneath a pile of sheets and blankets there was an injured
owl, my mom must have found it injured or injured it herself.

``Why do you have it in the trunk, mom?'' I said; keep in mind I don't remember
exactly the words used, or even if they're spoken at all; this is purely from
my memory.

``Because owls don't like the dark better than the day.'' She must have used an
example to compare the owl to, as that is her nature.

After a long time of travel, and a few more sleep sessions for me, we ended up
on a small dirt road surrounded by sagebrush, a familiar scene, with an
accompanying forest to our left; to our right was a magnificent black mountain
that looked like a mountain I had seen out of
{\color{blue}\underline{\href{https://www.history.com/shows/ancient-aliens}
{Ancient Aliens}}}, my father is a fan of the show, where there was some sort
of snake being with carvings on the walls and it would kill you if you got too
close; I don't remember the exact details. We turned left off of the dirt road
that led us to the area, my mom had the plan of going for a walk in the woods
and so we hopped the fence that separated the sagebrush from the forest, I
didn't want to and I expressed it, we hopped over the fence once again and went
back into my moms car. At this point we were facing the opposite direction of
the one that led us there and there was a little building or something along
the border fence with it's area squared off with a gate, the stereotypical farm
gate that everyone is familiar with; it was facing us and it was going to be a
bit of a project to get out the way we came without driving over the sagebrush.
As my mom was beginning to go through the struggle of backing out I told her of
the mountain I'd seen in Ancient Aliens, apparently it freaked her out because
she just rammed through the gate and drove out that way; the building was
accompanied by two of these gates but the second one was open, the one that led
the rest of the way to the dirt road. While I was associating the nearby
mountain with the one I saw on Ancient Aliens, I genuinely had the plan of
telling her; I don't remember if it was to get her to leave the area or because
I was scared of it, but it accomplished the prior. Along with getting my mom to
leave the area, my elaboration on the mountain also lead her car to have a
broken windshield, a hole that went straight through the driver's side of the
windshield. She tried to cover it up with a piece of cardboard from the trunk
but it was no use, we drove around with a loud clapping noise every now and
then and negative cabin pressure.

Eventually my mom got pulled over, my mother answered the cops' questions with a
fear in her voice, he questioned the hole in the windshield and the damage to
the front end, I don't remember what her answer was but it's evident that it
wasn't convincing because I soon found myself in the passenger seat of a cop car
on my way to an orphanage. The cop got wildlife services to obtain the owl. On
the way to the orphanage we drove passed a frozen over lake, opposing lanes
branded around it; there was a car that had falling into the lake, it was upside
down with another car, it must have been a pretty extreme accident. The cop
found a way to drive onto the ice, winter must be so severe there that one can
feel comfortable driving on a frozen lake (I believe we were in Idaho, but again
I don't feel the need to ask my parents for exact details), he stayed on scene
helping as much as he could until other cops arrived, he had radioed in that he
had a kid with him that had to get to the orphanage. I don't remember saying
anything to the cop, but I was smart enough that I must have given him
information about my father, we must have talked about the owl and life in
general, as I have vague images that relate to those concepts.

The orphanage was depressing, I was already familiar with what life was like at
one of them; the caretakers believed that I was to stay there until adulthood
and thus I believed the same. Part of the coping process I went through involved
visiting everyone's room; one person, an older boy probably around fifteen years
old, had a TV that was tuned to a channel playing
{\color{blue}\underline{\href{https://www.britannica.com/topic/Tom-and-Jerry}{Tom
and Jerry}}}, one of my all-time favorite shows, this helped me to accept my
fate. I arrived at the orphanage a few hours before nightfall; around the time
eventide came I was told that my dad was coming to get me. He arrived near
midnight on my sisters car, a hybrid Toyota Corolla that I don't know the year
of, with my brother in law James Mondragon and two others. We drove home in the
dark and I got the best sleep I had gotten since I left my home.

After that incident my mom ended up in a mental hospital for a while. I remember
calling her and talking about words and language, discussing happenings in
school, and asking her for advice; she was, and still is, in a healthy state of
mind. My mom has a deep understanding of physics phenomena and spiritual and
physical wellbeing. Truly I'm not fully aware of the situation surrounding me,
my intuition gives me a dull concept of my family, but I still find a way to
love them, everything has led up to this point and I'm ready to move forward
with the biggest punch I can muster.

\subsection*{Hola}

\end{document}
