\documentclass[12pt]{article}
\usepackage{titling}
\usepackage{setspace}
\usepackage{hyperref}
\usepackage{lipsum}
\usepackage[shortlabels]{enumitem}
\usepackage[margin=1in]{geometry}

\newcommand{\PutTitle}[1]
{
    \begin{center}
        {\huge\bfseries\thetitle}\\
        by \theauthor\\
        \thedate\\
        #1        
    \end{center}
    \hrule
    \vspace{2ex}
}

\setlength\paperwidth{8.5in}
\setlength\paperheight{11in}
\setlength\parindent{24pt}

\hypersetup
{
    colorlinks=true,
    linkcolor=blue,
    urlcolor=blue,
}

\begin{document}

\title{Hammurabi's Code}
\author{Jonah Mondragon}
\date{\today}
\PutTitle{World History Period 7}

\doublespacing

\section{Document A: Religion}

\begin{enumerate}
    \item{According to this document, where did Hammurabi get his power as
    king?}
    \paragraph~
    According to the document, Anu and Bel are who gave power to Hammurabi as
    king, the test implies that they are gods, Bel being the lord of Heaven, and
    Anu is implied to also be a god through association with Bel.
 
    \item{Monotheistic or Polytheistic?}
    \begin{enumerate}[1.]
        \item{According to this document, was Babylonia a monotheistic society
        (belief in one god) or a polytheistic society (belief in many gods)?}
        \paragraph~
        Polytheism.

        \item{How do you know this from Hammurabi’s Code?}
        \paragraph~
        ``Source: `Code of Hammurabi,' 1780 BCE.''        
    \end{enumerate}

    \item{According to this document, what is the goal of Hammurabi’s Code?}
    \paragraph~
    The goal of Hammurabi's code is the ``bring about the rule of righteousness.''
\end{enumerate}

\section*{Document B: Economy}

\begin{enumerate}

\item{Working the fields: Summarize laws 42-43 in your own words.}
\begin{enumerate}[42.]
    \item{When borrowing land for the use of cultivation, the borrower must
    still pay the rent for the land even if no crops were grown.}
\end{enumerate}
\begin{enumerate}[43.]
    \item{When borrowing land for cultivation, and the borrower let's the soil become
    unfertile, he shall grow}
\end{enumerate}

\item{The dams: Summarize laws 53-54 in your own words.}
\begin{enumerate}[53.]
    \item{Hello}
\end{enumerate}
\begin{enumerate}[54.]
    \item{Hello}
\end{enumerate}

    \item{Type of Economy}
    \begin{enumerate}[1.]
        \item{According to this document, do you think most people in Babylonia made
        money in cities or in the country?}
        \item{How do you know this from Hammurabi’s Code?}
        \paragraph~
        ``Source: `Code of Hammurabi,' 1780 BCE.''        

    \end{enumerate}
\end{enumerate}

\section{Document C: Society}

 

\begin{enumerate}
\item{Equality}
    \begin{enumerate}[1.]
        \item{Code 196: What is the punishment for putting out the eye of “another
            man”?}

        \item{How might code 196 be seen as an attempt to promote “equality”?}

        \item{Code 199 describes a different punishment for putting out the eye of
            an enslaved What might this suggest about equality in Babylonia?}

        \item{Find two other passages that provide evidence of inequality in
        Hammurabi’s (Be sure to cite the number of the code.)}
        \begin{enumerate}[1.]
            \item{Evidence 1:}
            \item{Evidence 2:}
        \end{enumerate}
    \end{enumerate}
 
\item{What do codes 117, 138, & 141 suggest about the status of women in
Babylonian society?}



\end{enumerate}

\end{document}

