\documentclass[12pt]{article}
\usepackage{titling}
\usepackage{setspace}
\usepackage{hyperref}
\usepackage{lipsum}
\usepackage[shortlabels]{enumitem}
\usepackage[margin=1in]{geometry}

\newcommand{\PutTitle}[1]
{
    \begin{center}
        {\huge\bfseries\thetitle}\\
        by \theauthor\\
        \thedate\\
        #1        
    \end{center}
    \hrule
    \vspace{2ex}
}

\setlength\paperwidth{8.5in}
\setlength\paperheight{11in}
\setlength\parindent{24pt}

\hypersetup
{
    colorlinks=true,
    linkcolor=blue,
    urlcolor=blue,
}

\begin{document}

\title{Hammurabi's Code}
\author{Jonah Mondragon}
\date{\today}
\PutTitle{World History Period 7}

\doublespacing

\section{Document A: Religion}

\begin{enumerate}
    \item{According to this document, where did Hammurabi get his power as
    king?}
    \paragraph~
    According to the document, Anu and Bel are who gave power to Hammurabi as
    king, the test implies that they are gods, Bel being the lord of Heaven, and
    Anu is implied to also be a god through association with Bel.
 
    \item{Monotheistic or Polytheistic?}
    \begin{enumerate}[1.]
        \item{According to this document, was Babylonia a monotheistic society
        (belief in one god) or a polytheistic society (belief in many gods)?}
        \paragraph~
        Polytheism.

        \item{How do you know this from Hammurabi’s Code?}
        \paragraph~
        ``Source: `Code of Hammurabi,' 1780 BCE.''        
    \end{enumerate}

    \item{According to this document, what is the goal of Hammurabi’s Code?}
    \paragraph~

    The goal of Hammurabi's code is the ``bring about the rule of
    righteousness.'' \end{enumerate}

\section*{Document B: Economy}
\begin{enumerate}
\item{Working the fields: Summarize laws 42-43 in your own words.}
\begin{enumerate}[42.]
    \item{When borrowing land for the use of cultivation, the borrower must
    still pay the rent for the land even if no crops were grown.}
\end{enumerate}
\begin{enumerate}[43.]
    \item{When borrowing land for cultivation, and the borrower let's the soil
    become unfertile, he shall grow}
\end{enumerate}

\item{The dams: Summarize laws 53-54 in your own words.}
\begin{enumerate}[53.]
    \item{If one fails to maintain their dam and their neglect leads to the dam
    breaking, they are responsibile for compensating any of the crops lost in
    others' fields.}
\end{enumerate}
\begin{enumerate}[54.]
    \item{If one who's dam breaks and damages others' crops cannot compensate
    the others, then him and all of his processions must be divided among those
    affected.}
\end{enumerate}

    \item{Type of Economy}
    \begin{enumerate}[1.]
        \item{According to this document, do you think most people in Babylonia
        made money in cities or in the country?}
        \paragraph~
        The codes of Hammurabi shown in the document all involve the logistics
        of agricultural land and the distribution and development of crops;
        therefore I'd imagine money came in the form of grain and most of the
        money was made from the country areas.

        \item{How do you know this from Hammurabi’s Code?}
        \paragraph~
        ``Source: `Code of Hammurabi,' 1780 BCE.''        

    \end{enumerate}
\end{enumerate}

\section{Document C: Society}
\begin{enumerate}
\item{Equality}
    \begin{enumerate}[1.]
        \item{Code 196: What is the punishment for putting out the eye of
        ``another man?''}
        \paragraph~
        The eye of the man the put out the eye of the other, shall also have his
        eye put out.

        \item{How might code 196 be seen as an attempt to promote “equality”?}
        \paragraph~
        Code 196 may be considered an attempt to promote equality as by the
        decree of the code, any man who wrongfully afflicts another will have
        the same wrongful affliction done upon them in return, and it doesn't
        discuss what would occur based on social status.

        \item{Code 199 describes a different punishment for putting out the eye
        of an enslaved. What might this suggest about equality in Babylonia?}
        \paragraph~
        Those that were considered slaves were only considered at the value of
        money, as the only punishment for wrongfully afflicting upon another's
        slave is to pay of the slave's value.

        \item{Find two other passages that provide evidence of inequality in
        Hammurabi’s (Be sure to cite the number of the code.)}
        \begin{enumerate}[1.]
            \item{Code 202 stats that if one strikes the body of one above him,
            then he shall get whipped with an ox whip sixty times; this is
            evidence of inequality as, in conjecture, it wouldn't work the other
            way around.}

            \item{Code 203 states that if one strikes another free-born man of
            equal rank, that he only needs to pay one gold mina, whatever that
            is, which implies that if the man that was stricken was below is
            rank, or of slave status, there would be no punishment.}
        \end{enumerate}
    \end{enumerate}
 
\item{What do codes 117, 138, \& 141 suggest about the status of women in
Babylonian society?}

\paragraph~
Based on the Hammurabi codes 117, 138 and 141, women had a lower social status
then men, as well as children; they can be sold for labor, even though the time
of labor is limited to only three years; a man doesn't quarrel about a divorce
from a woman when not married; marriage is only broken because of the wish of
the wife if the husband allows it, if he doesn't and he takes another then the
first wife must stay as a servant to the husband.

\end{enumerate}

\end{document}

