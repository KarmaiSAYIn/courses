\documentclass{article}
\usepackage[hyphens]{url}
\usepackage{hyperref}
\usepackage{color}
\usepackage[margin=1in]{geometry}

\setlength\paperwidth{8.5in}
\setlength\paperheight{11in}

\hypersetup
{
    colorlinks=true,
    urlcolor=blue,
}

\input{~/.macros}

\title{5 Facts of 5 Historical Events in Taos}
\author{Jonah Mondragon}
\date{\today}

\begin{document}

\PutTitle{New Mexico History Period 3}
\pagestyle{plain}

\section{The Attack on Columbus}
\begin{enumerate}
    \item{The Attack on Columbus wasn't against Christopher Columbus, as you may think, but on the town of Columbus, New Mexico.}
    \item{This attack was initiated by {\color{blue}\underline{\href{https://www.history.com/topics/mexico/pancho-villa}{Pancho Villa}}} because of his grudge at the fact of Carranza's quick surrender of, what is now, New Mexico to the United States.}
    \item{Another inspiration for this attack was a burning of around 20 Mexicans after their arrest; which happened just two days before.}
    \item{The attack occurred at 4:45AM on March 9, 1916; Pancho Villa didn't even participate in the attack, he stayed on the Mexican side of the border.}
    \item{Pancho's men retreated with several wounded a few hours later; rather objectively pointless.}
\end{enumerate}

\section{New Mexico and World War I}
\begin{enumerate}
    \item{The United States entered World War I started on April 2, 1917.}
    \item{The reason for the United States' entry into the war was because of a German sinking of a British ship with American civilians in it.}
    \item{A third of New Mexican soldiers volunteered for the war.}\label{TEST}
    \item{The involvment of New Mexico on the war was is attributed to Pancho Villa, and his raid on Columbus, New Mexico; which provided the U.S. military with experience.}
    \item{It was a New Mexican that lead the first American night reconnaissance mission.}
\end{enumerate}

\section{The Spanish Flu in New Mexico}
\begin{enumerate}
    \item{The Spanish flu started immediately following the end of World War I, in 1918.}
    \item{Fifteen thousand New Mexicans were affected by the Spanish flu, and over a thousand died from it.}
    \item{During the Spanish flu, New Mexico didn't have a Department of Health.}
    \item{The Native Americans in New Mexico were especially affected by the flue.}
    \item{After the Spanish flu, a Department of Health was initialized.}
\end{enumerate}

\newpage

\section*{Sources}
\begin{enumerate}
    \item{\url{https://www.laits.utexas.edu/jaime/jrn/cwp/pvg/columbus.html}}
    \item{\url{https://www.history.com/topics/mexico/pancho-villa}}
    \item{\url{https://www.abqjournal.com/1091131/forgotten-sacrifice-historians-aim-to-keep-memory-of-world-war-i-alive.html}}
    \item{\url{https://newmexicohistory.org/2013/11/02/influenza-epidemic-in-new-mexico-1918/}}
\end{enumerate}

\end{document}
